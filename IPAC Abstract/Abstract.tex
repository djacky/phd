\documentclass[11pt,english]{article}
\usepackage[T1]{fontenc}
\usepackage{babel}
\bibliographystyle{IEEEtran}

\author{
  Achille Nicoletti and Michele Martino\\ \\
  \textit{\small The European Organization for Nuclear Research (CERN)} \\
  \textit{\small Technology Department, Electrical Power Converters group,} \\
  \textit{\small High Precision Measurements section}
}

\title{Data-driven Controller Design for a Fast-Pulsed Power Converter Control Application}
\date{}

\begin{document}
  \maketitle
  
  \begin{center}
  \LARGE{\bf{Abstract}}
  \end{center}
 
The complex structures of today's systems create challenging tasks for the control engineer in designing a viable controller. In order to simplify the controller design process, these systems are approximated with low-order models; this reduces both time and effort in synthesizing a controller. However, this approximation can create stability and performance problems since these low-order models are subject to model uncertainty. Data-driven control methods seek to alleviate this problem by synthesizing controllers based on time-domain or frequency-domain data (i.e., synthesis is model independant). A survey on the differences associated with model-based control and data-driven control has been addressed in \cite{HW13} and \cite{BCE12}; the authors assert that model-based control methods are inherently less robust due to the unmodeled dynamics of a process, and that these controllers are unsafe for practical applications. In other words, the parametric uncertainties and the unmodeled dynamics associated with the data-driven scheme are irrelevant, and the only source of uncertainty is the measurement noise.

Data-driven control methods using frequency-domain data are design schemes that continue to spark the interest of many researchers. The frequency-domain approach offers many advantages compared to time-domain methods :
\begin{itemize}
\item Without knowledge of the transfer function, the dynamics of a system can be captured experimentally through the frequency response.
\item Relative and absolute stability of a closed-loop system can be determined with the knowledge of the open-loop frequency response.
\item Noise disturbance generated in the system can be easily determined using frequency analysis.
\item System uncertainties can be modeled at discrete frequency points (which reduces the conservatism associated with uncertainty modeling). 
\end{itemize}

For this specific application, an $RST$ controller structure was implemented to control the dynamics of the power converter control system. The $RST$ controller structure is an effective discrete-time two-degree of freedom (2DOF) polynomial controller where the tracking and regulation characteristics of a closed-loop system can be formulated independently \cite{LZ06}. To design this controller, a $\mathcal{H}_\infty$ loop-shaping method was used to shape a desired sensitivity function. Robust controller design methods belonging to the $\mathcal{H}_{\infty}$ control framework minimizes the $\mathcal{H}_{\infty}$ norm of a weighted closed-loop sensitivity function \cite{ZD98}. In this method, a convex optimization problem can be formulated if each of the $RST$ polynomials are linearly parameterized \cite{NEK15}. 

For this power converter application, it was desired to track a desired reference signal while maintaining sufficient stability margins. Therefore, 


FU biatch!
\bibliography{linear}
\end{document}